\documentclass[chapter,kosection, 10.5pt, romanfixed, a4paper]{oblivoir}
\usepackage{fapapersize}
\usefapapersize{210mm,297mm,10mm,*,10mm,10mm}
\usepackage[usenames,dvipsnames]{color}
\usepackage[pdftex]{graphicx}
\usepackage{fancybox}

\SelectHfonts{utbm,utbm,utpg}{utyt,utgt,utgt}
\SetHangulspace{1.5}{1.2}
\addtolength{\droptitle}{-8ex}

\newenvironment{smallquote}
{\begin{quote}\small} {\end{quote}}


\begin{document}

\title{프로그래밍 온라인 게임}
\author{darkface}
\maketitle

\begin{quote}
\flushright \footnotesize
컴퓨터 프로그래밍은 시나 음악의 창작과 같은 예술의 형식이다. \\
- 도날드 크누스
\end{quote}
\qquad

\tableofcontents

\oblivoirchapterstyle{openright}
\chapter{긴 여정의 시작}

이제 프로그래머로서 온라인 게임을 만드는 긴 과정으로 들어가 봅니다. 그 간의 경험이 일천하여 
정리될 내용이 있을까 싶기도 하지만 스스로 공부하는 삼아 추가적인 테스트와 코드를 겸하여 
진행하고자 합니다. 

이 책을 쓰면서 전에 작성한 네트워크 엔진을 시뮬레이션을 구현한 서버 엔진으로 발전시키고 
테스트를 겸해서 클라이언트 엔진을 하나 만들려고 합니다. 여러 종류의 게임을 잘 만들고 싶고 
전체 구조가 안정적이어서 프로그래머는 고생하지 않고 기획자는 만족하고 그래픽은 펑펑 내용을 
쏟아 내고 사용자는 행복한 기술 기반을 만들려고 하는 게 제 남은 인생 동안 가려고 하는 길입니다. 
\\
\\
\textcolor{RoyalBlue}{자 이제 긴 여행을 시작해 봅시다.}

\section{게임 구조의 이해}

게임은 컴퓨터에서 창조된 세계를 표현합니다. 텍스트, 이미지, 사운드 등이 최종적인 
표현의 산출물입니다. 따라서, 게임을 만들 때는 창조된 세계와 표현 두 가지를 구현해야 합니다. 

창조된 세계는 가상 현실과 비슷한 개념이지만 상상력 만으로 제한 받는 세계이기 때문에 
더욱 넓은 개념이고 어찌 보면 끝이 없기 때문에 게임 개발을 매력적으로 만드는 이유입니다. 

\subsection{창조된 세계}

\shadowbox{창조된 세계는 오브젝트와 오브젝트 간의 상호 작용의 집합이다}

오브젝트는 마땅히 정의할 말이 없습니다. 우리가 있다고 믿는 어떤 것인데 결국은 상호 작용을 
통해서만 존재가 드러납니다. 하지만 일정하게 반응하기 때문에 우리가 알게 되는 것들입니다. 
상호 작용이 존재 자체의 근거가 된다는 점은 양자 역학의 불확정성 원리로도 알 수 있습니다. 

\begin{smallquote}
불확정성원리(Uncertainty principle) 는 양자 역학에서 맞바꿈 관측량(commuting observables)이 아닌 두 개의 
관측가능량(observable)을 동시에 측정할 때, 둘 사이의 정확도에는 물리적 한계가 있다는 원리이다. 
불확정성 원리는 양자역학에 대한 추가적인 가정이 아니고 양자역학의 통계적 해석으로부터 얻어진 근본적인 결과이다. 
하이젠베르크의 불확정성원리는 위치-운동량에 대한 불확정성원리이며, 입자의 위치와 운동량을 동시에 정확히 
측정할 수 없다는 것을 뜻한다. 위치가 정확하게 측정될 수록 운동량의 퍼짐(또는 불확정도)는 커지게 되고 
반대로 운동량이 정확하게 측정될 수록 위치의 불확정도는 커지게 된다.
\end{smallquote}

결국 창조된 세계에서 가장 중요한 것은 상호 작용입니다. 이를 통해서만 경험할 수 있고 믿을 수 있고 
뭔가가 있다고 생각합니다. 

\subsection{표현}

상호 작용은 관측될 수 있어야 합니다. 표현되지 않으면 알 수 없기 때문입니다. 표현은 게임에서 
더 중요한 작용을 합니다. 재미있다고 생각하고 현실처럼 생각합니다. 즉, 몰입할 수 있게 해줍니다. 

표현은 아직 오감 중 시각과 청각이 주가 되고 키보드나 마우스를 사용하면서 일부 촉각을 사용합니다. 
청각 표현은 이미 어느 정도 원하는 만큼 표현 가능한 수준에 도달했습니다. 하지만 아직 시각적인 표현은 
무궁무진하게 남아 있습니다. 물론 청각도 언제 어떻게 표현하냐에 따라 정말 새로운 느낌의 세계를 
만들 수 있습니다. 소리만으로 진행되는 게임도 가능하겠지요. 옛날 라디오 드라마 처럼요. 

\subsubsection{사실감}

\subsubsection{일관성}

\subsubsection{몰입}


\subsection{상호작용}

이제 게임은 어떤 상호작용이 가능한가와 이를 어떻게 표현하는가가 가장 중요한 기술과 기획적인 요소가 된다는
사실을 알았습니다. 영화처럼 그림만 보여주는 이야기 방식과 게임을 근본적으로 다르게 만든다는 점에서 
상호작용이 가장 중요합니다. 

\shadowbox{게임은 상호작용과 표현의 구조이다}

게임 기획자들은 저처럼 생각하지는 않을 겁니다. 그런 기획자는 혼나야 하겠지요. 하지만 프로그래머라면
이렇게 근본이 되는 구조를 파악하고 이해해야 합니다. 





\end{document}